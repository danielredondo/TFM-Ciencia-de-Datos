\chapter{Introducción}

\section{Objetivos del trabajo}

En el presente Trabajo Fin de Máster se analiza la epidemiología de los cánceres de hígado y colon-recto y se detectan genes que permiten identificar tumores.

\begin{itemize}
	\item En el capítulo 1, 
	\item En el capítulo 2,
	\item En el capítulo 3,
	\item En el capítulo 4,
	\item En el capítulo 5,
	\item En el capítulo 6,
\end{itemize}

% ---------------------------------

\section{Cáncer}

El cáncer es una enfermedad en la que se produce una división incontrolada de las células \cite{AmericanCancerSociety2015}. Aunque generalmente se habla del cáncer como una única enfermedad se trata en realidad de un conjunto de enfermedades, existiendo más de 100 tipos distintos de cáncer \cite{NationalCancerInstitute2015}.\\

El cáncer es una enfermedad genética, esto es, causada por cambios en los genes que controlan las funciones celulares \cite{NationalCancerInstitute2015}. En general, el proceso de creación del cáncer es complejo y multifactorial: a menudo el causante no es un solo elemento, sino la combinación e interacción de distintos factores ambientales y genéticos \cite{Migliore2012}.\\

Los factores causantes del cáncer se pueden clasificar principalmente en tres categorías:
\begin{enumerate}
	\item Factores no modificables. Son elementos que no se pueden cambiar, como la edad o la herencia genética \cite{WorldHealthOrganization2014, WorldHealthOrganization2020}.
	\item Factores modificables o prevenibles, como el tabaco, el alcohol, la dieta o la exposición a distintos carcinógenos \cite{Cogliano2011}.
	\item Otros factores. Algunas circunstancias no se corresponden a ninguna de las categorías anteriores ya que algunos de sus aspectos no se pueden cambiar. Es el caso de  factores socioeconómicos (como cobertura sanitaria en el lugar de residencia o privación económica) y factores reproductivos u hormonales (como toma de anticonceptivos, lactancia materna o terapia hormonal sustitutiva en mujeres menopáusicas) \cite{WorldHealthOrganization2020}.
\end{enumerate}

A continuación se introducen dos tipos de cáncer con los que se trabajará más adelante: el cáncer de hígado y el cáncer de colon-recto.

% ---------------------------------

\subsection{Cáncer de hígado}

El cáncer de hígado se corresponde con el código C22 de la Clasificación Internacional de Enfermedades, Décima Revisión, integrando las neoplasias malignas de hígado y vías biliares intrahepáticas \cite{ICD10, cie10es}.

\subsubsection{Anatomía y funciones del hígado}

El hígado es el órgano interno más grande y pesado del cuerpo humano, está situado en el cuadrante superior derecho del abdomen, debajo de las costillas, y está compuesto principalmente por dos lóbulos \cite{Abdel-Misih2010}.\\

\newpage
\textbf{Figura 1}. Anatomía del abdomen humano. Ilustración de Ties van Brussel.
\begin{center}
\includegraphics[width=.70\textwidth]{figuras/anatomia_higado.png} \\
\end{center}

Las funciones del hígado son múltiples y diversas. Las principales son procesar, particionar y metabolizar macronutrientes, regular el volumen de sangre, apoyar al sistema inmune, eliminar sustancias químicas como el alcohol y otras drogas y producir bilis para absorber grasas \cite{Trefts2017}. Es un órgano imprescindible para la vida.

\subsubsection{Factores de riesgo}

Uno de los factores de riesgo más comunes del cáncer de hígado es la presencia de cirrosis, o sustitución de células sanas de hígado por tejido cicatrizado. La cirrosis puede producirse por varias causas, siendo las más habituales el consumo excesivo de alcohol y la infección con el virus de la hepatitis B o C \cite{AmericanCancerSociety2019}. Otros factores de riesgo son el tabaco, la obesidad, padecer diabetes tipo II y consumir esteroides anabólicos \cite{AmericanCancerSociety2019, Marrero2005}.\\

La prevención del cáncer de hígado se basa en reducir la exposición a factores de riesgo como el tabaco y el alcohol, y en vacunarse contra la hepatitis B \cite{AmericanCancerSociety2019}.

% ---------------------------------

\subsection{Cáncer de colon-recto}

Las neoplasias malignas de colon, recto, unión rectosigmoidea, ano y canal anal (códigos C18-C21 según la Clasificación Internacional de Enfermedades, Décima Revisión \cite{ICD10, cie10es}) a menudo se estudian agrupadas por tener características muy similares.

\subsubsection{Anatomía y funciones del colon-recto}

El colon tiene 3 funciones principales: absorción de agua y electrolitos, producción y absorción de vitaminas y movimiento de heces hacia el recto para su eliminación por el ano \cite{Azzouz2020}.\\

\textbf{Figura 2}. Anatomía del intestino humano. Ilustración de Ties van Brussel.
\begin{center}
	\includegraphics[width=.70\textwidth]{figuras/anatomia_cr.png} \\
\end{center}

\subsubsection{Factores de riesgo}

Entre los factores de riesgo del cáncer de colon-recto se puede distinguir entre factores modificables y no modificables.\\

Entre los factores de riesgo que son modificables destacan el sobrepeso, la inactividad física, las dietas con alto consumo de carnes rojas o procesadas, y el consumo de tabaco y alcohol \cite{AmericanCancerSociety2020}.\\

Una edad superior a 50 años, padecer diabetes tipo 2 y tener antecedentes personales o familiares de cáncer de colon-recto, pólipos o enfermedad intestinal inflamatoria, como colitis ulcerosa y enfermedad de Crohn, son algunos de los factores de riesgo no modificables \cite{AmericanCancerSociety2020}. También existen algunos síndromes hereditarios como el síndrome de Lynch que aumentan las posibilidades de padecer cáncer de colon-recto \cite{Lynch2003}.\\

Para intentar prevenir el cáncer de colon-recto se deben cambiar aquellos factores que son modificables: realizar ejercicio, mantener una dieta saludable y evitar el consumo de tabaco y alcohol. Además, en los últimos años se están implementando programas de cribado de cáncer de colon-recto para detectar pólipos o diagnosticar el cáncer en etapas iniciales mediante análisis como pruebas de sangre oculta en heces o colonoscopias \cite{Levin2008}.\\

% ---------------------------------

\section{Ciencias -ómicas}

Se presenta a continuación una corta introducción a las ciencias -ómicas, con el objetivo de comprender los conceptos que se utilizarán más adelante.

\subsection{Algunas definiciones básicas}

\begin{itemize}
	\item Los seres vivos están hechos de células. En el núcleo de cada célula se encuentran los cromosomas, estructuras que almacenan el material genético del individuo. Las células humanas tienen 46 cromosomas: 23 heredados de la madre y 23 heredados del padre.
	\item La información genética se transporta mediante los ácidos nucleicos: ácido desoxirribonucleico (DNA, por sus siglas en inglés) y ácido ribonucleico (RNA, por sus siglas en inglés) \cite{Pierce2010}.
	\item En el DNA hay 4 tipos de bases nitrogenadas: A, C, G y T.
	\item En el RNA hay 4 tipos de bases nitrogenadas: A, C, G y U.
	\item Los cromosomas están formados por ácido desoxirribonucleico (DNA, por sus siglas en inglés), una molécula que codifica las instrucciones genéticas para la vida.
	\item Un gen es la región del DNA que codifica una proteína. Las proteínas son cadenas de aminoácidos unidos por enlaces peptídicos (enlaces entre el grupo amino y carboxilo).
	\item El genoma es la secuencia de nucleótidos que forman el ADN de un individuo.
	\item El ácido ribonucleico (RNA, por sus siglas en inglés) es un ácido nucleico formado por ribonucleótidos. 
\end{itemize}

\subsection{Genómica}

La genómica es la ciencia que estudia la composición, estructura y función de los genomas. Se dedica por tanto a estudiar cromosomas, mutaciones y variaciones tanto de nucleótidos concretos como de regiones del genoma.\\

No debe confundirse con la genética, que estudia los genes de manera individual.\\

El análisis GWAS (Genome-wide association study) es un ejemplo de análisis genómico.

\subsection{Transcriptómica}

La transcriptómica estudia el transcriptoma, esto es, el conjunto de RNA presente en una célula). El transcriptoma indica el nivel de expresión de genes en un determinado momento.\\

Los análisis de RNA-Seq y microRNA se enmarcan en el ámbito de la transcriptómica. \\

\subsection{Otras ciencias -ómicas}

La proteómica es la ciencia que estudia y caracteriza el proteoma (imagen dinámica de todas las proteínas expresadas).\\

\textcolor{red}{Metabolómica}.

\section{RNA-Seq}

Leer \cite{Stark2019, VanVerk2013, CastilloSecilla2020}.\\

Este trabajo se enmarca dentro de la transcriptómica, y está basado en datos obtenidos mediante RNA-Seq, técnica en la que se cuentan distintas lecturas de cada gen para ver si están sobreexpresados o infraexpresados, para finalmente comparar esas expresiones con una referencia (por ejemplo, enfermos contra sanos). Como sólo se realizan cuentas de la expresión de los genes, el RNA-Seq de un individuo no permite su identificación, por lo que los datos a menudo son accesibles de manera abierta.
