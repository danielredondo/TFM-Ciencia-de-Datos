\chapter{Epidemiología del cáncer}

Definición epidemiología \cite{IARC1999}.

\section{Indicadores epidemiológicos}

Para medir en la población el impacto del cáncer se utilizan principalmente cuatro indicadores:

\begin{itemize}
	\item \textbf{Incidencia} (casos nuevos). Mide el riesgo de presentar cáncer.
	\item \textbf{Mortalidad} (defunciones). Mide el riesgo de morir por cáncer.
	\item \textbf{Supervivencia} (porcentaje de casos vivos). Mide la historia natural del cáncer y efectividad del tratamiento.
	\item \textbf{Prevalencia} (casos nuevos y antiguos, vivos). Mide la carga asistencial de la enfermedad.
\end{itemize}

\textcolor{red}{Añadir tendencias}

% ---------------------------------

\textbf{Referencias:}

GLOBOCAN - \cite{Bray2018, GCO}

ECIS - \cite{ECIS, ECIS2}

REDECAN - \cite{REDECAN2020}

Población INE - \cite{INEpob}

Defunciones Ministerio - \cite{MSCBS}.

\section{Incidencia de cáncer}

\textcolor{red}{Cómo se obtiene la incidencia. Importancia de RCPoblacionales, estimaciones y proyecciones.}

\textcolor{red}{Número de casos. Tiene problemas por tamaño de población. Tasa bruta. Tiene problemas por estructura de población. Tasas estandarizadas (mundiales, europeas viejas y nuevas). Se usan a veces otros indicadores como tasas acumulativas.}

Las poblaciones estándar más utilizadas son:
\begin{itemize}
	
	
	\item Antigua población estándar europea. Propuesta en 1976 \cite{Waterhouse1976} basándose en la estructura de edad de varias poblaciones escandinavas.
	
	\item Nueva población estándar europea. En el año 2013 se realiza una revisión de la población estándar europea de 1976 por parte de la Oficina Europea de Estadística (EUROSTAT) para que la población refleje fielmente el envejecimiento existente en la población europea \cite{EUROSTAT2013}. Debido a su novedad, el uso de esta población aún no está ampliamente extendido en los organismos internacionales \cite{ECIS2} y en ocasiones se reportan las dos tasas estandarizadas por las poblaciones estándar antigua y nueva \cite{ECIS}.
	
	\item Población mundial. Propuesta por primera vez en 1960 \cite{SegiM.1960} y modificada más tarde en 1966 \cite{Doll1966}, permite realizar comparaciones a nivel mundial.
\end{itemize}

\newpage
\textbf{Tabla 1}. Poblaciones estándar más frecuentes.
\begin{table}[H]
	\begin{tabular}{|c|c|c|c|}
		\hline
		
		Grupo de edad  &  \begin{tabular}[c]{@{}c@{}}Población estándar\\ mundial\end{tabular}  &  \begin{tabular}[c]{@{}c@{}}Población estándar\\ europea 1976\end{tabular}  &  \begin{tabular}[c]{@{}c@{}}Población estándar\\ europea 2013\end{tabular}\\\hline
		
		0-4 años  &  12.000  &  8.000  &  5.000\\
		5-9 años  &  10.000  &  7.000  &  5.500\\
		10-14 años  &  9.000  &  7.000  &  5.500\\
		15-19 años  &  9.000  &  7.000  &  5.500\\
		20-24 años  &  8.000  &  7.000  &  6.000\\
		25-29 años  &  8.000  &  7.000  &  6.000\\
		30-34 años  &  6.000  &  7.000  &  6.500\\
		35-39 años  &  6.000  &  7.000  &  7.000\\
		40-44 años  &  6.000  &  7.000  &  7.000\\
		45-49 años  &  6.000  &  7.000  &  7.000\\
		50-54 años  &  5.000  &  7.000  &  7.000\\
		55-59 años  &  4.000  &  6.000  &  6.500\\
		60-64 años  &  4.000  &  5.000  &  6.000\\
		65-69 años  &  3.000  &  4.000  &  5.500\\
		70-74 años  &  2.000  &  3.000  &  5.000\\
		75-79 años  &  1.000  &  2.000  &  4.000\\
		80-84 años  &  500  &  1.000  &  2.500\\
		$\geq$85 años  &  500  &  1.000  &  2.500\\\hline
		
	\end{tabular}
\end{table}


\noindent Se notará ASR-W (de \textit{world}) a la tasa estandarizada por la población mundial, ASR-E a la tasa estandarizada por la población europea de 1976 y ASR-nE a la de 2013.\\

\subsection{Incidencia del total del cáncer excepto piel no melanoma}

\textcolor{red}{Poca importancia de piel no melanoma}\\

\textcolor{red}{Diagrama de Marimekko de incidencia de cáncer. Añadir categoría de Otros}

\subsection{Incidencia de cáncer de hígado}

texto

\subsection{Incidencia de cáncer de colon-recto}

texto

\section{Mortalidad por cáncer}

\textcolor{red}{Cómo se obtiene la mortalidad. Importancia de certif de defunción. También estimaciones y proyecciones.}

\subsection{Mortalidad del total del cáncer excepto piel no melanoma}

texto

\subsection{Mortalidad de cáncer de hígado}

texto

\subsection{Mortalidad de cáncer de colon-recto}

texto

\section{Supervivencia de cáncer} 

\textcolor{red}{Supervivencia se calcula principalmente a partir de inc, mort y tablas de vida población general}

\subsection{Supervivencia del total del cáncer excepto piel no melanoma}

texto

\subsection{Supervivencia de cáncer de hígado}

texto

\subsection{Supervivencia de cáncer de colon-recto}

texto

\section{Prevalencia de cáncer}

texto

\subsection{Prevalencia del total del cáncer excepto piel no melanoma}

texto

\subsection{Prevalencia de cáncer de hígado}

texto

\subsection{Prevalencia de cáncer de colon-recto}

texto






