\chapter{Epidemiología del cáncer de páncreas}

\section{Indicadores epidemiológicos}

Para medir en la población el impacto del cáncer se utilizan principalmente cuatro indicadores:

\begin{itemize}
	\item \textbf{Incidencia} (casos nuevos). Mide el riesgo de presentar cáncer.
	\item \textbf{Mortalidad} (defunciones). Mide el riesgo de morir por cáncer.
	\item \textbf{Supervivencia} (porcentaje de casos vivos). Mide la historia natural del cáncer y efectividad del tratamiento.
	\item \textbf{Prevalencia} (casos nuevos y antiguos, vivos). Mide la carga asistencial de la enfermedad.
\end{itemize}

\textcolor{red}{AÑADIR TENDENCIAS!! DE TODO}

\section{Incidencia de cáncer de páncreas}

\subsection{Medidas usadas}

\textbf{Número de casos}

El número de casos es la medida más básica que se puede utilizar.

\textbf{Tasas brutas y tasas específicas por edad}

\textbf{Tasas estandarizadas por una población de referencia}

\textbf{Tasas acumulativas}

\textcolor{red}{Cómo se obtiene la incidencia. Importancia de RCPoblacionales, estimaciones y proyecciones.}

Para incidencia, la unidad de análisis es el tumor, no la persona.

\subsection{Incidencia de cáncer de páncreas en el mundo}

\subsection{Incidencia de cáncer de páncreas en Europa}

\subsection{Incidencia de cáncer de páncreas en España}

\subsection{Incidencia de cáncer de páncreas en Andalucía y Granada}


\section{Mortalidad por cáncer de páncreas}

\textcolor{red}{Cómo se obtiene la mortalidad. Importancia de certif de defunción. También estimaciones y proyecciones.}

\subsection{Mortalidad por cáncer de páncreas en el mundo}

\subsection{Mortalidad por cáncer de páncreas en Europa}

\subsection{Mortalidad por cáncer de páncreas en España}

\subsection{Mortalidad por cáncer de páncreas en Andalucía y Granada}

\section{Supervivencia de cáncer de páncreas} 

\textcolor{red}{Supervivencia se calcula principalmente a partir de inc, mort y tablas de vida población general}

\section{Prevalencia de cáncer de páncreas}

