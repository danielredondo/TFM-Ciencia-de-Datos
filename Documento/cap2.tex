\chapter{Epidemiología del cáncer}

Definición epidemiología \cite{IARC1999}.

\section{Indicadores epidemiológicos}

Para medir en la población el impacto del cáncer se utilizan principalmente cuatro indicadores:

\begin{itemize}
	\item \textbf{Incidencia} (casos nuevos). Mide el riesgo de presentar cáncer.
	\item \textbf{Mortalidad} (defunciones). Mide el riesgo de morir por cáncer.
	\item \textbf{Supervivencia} (porcentaje de casos vivos). Mide la historia natural del cáncer y efectividad del tratamiento.
	\item \textbf{Prevalencia} (casos nuevos y antiguos, vivos). Mide la carga asistencial de la enfermedad.
\end{itemize}

\textcolor{red}{Añadir tendencias}

% ---------------------------------

\section{Epidemiología del cáncer}

\textcolor{red}{Poca importancia de piel no melanoma}\\






\textcolor{red}{Medidas de incidencia}

\textcolor{red}{Cómo se obtiene la incidencia. Importancia de RCPoblacionales, estimaciones y proyecciones. Para incidencia, la unidad de análisis es del tumor, no la persona.}

\textcolor{red}{Número de casos. Tiene problemas por tamaño de población. Tasa bruta. Tiene problemas por estructura de población. Tasas estandarizadas (mundiales, europeas viejas y nuevas). Se usan a veces otros indicadores como tasas acumulativas.}


\textcolor{red}{Diagrama de Marimekko de incidencia de cáncer. Añadir categoría de Otros}

\textcolor{red}{Cómo se obtiene la mortalidad. Importancia de certif de defunción. También estimaciones y proyecciones.}

\textcolor{red}{Supervivencia se calcula principalmente a partir de inc, mort y tablas de vida población general}

\subsection{Incidencia de cáncer}

\subsection{Mortalidad por cáncer}

\subsection{Supervivencia de cáncer} 

\subsection{Prevalencia de cáncer}



% ---------------------------------

\section{Epidemiología del cáncer de hígado}

\subsection{Incidencia de cáncer de hígado}

GLOBOCAN - \cite{Bray2018, GCO}\\

ECIS - \cite{ECIS, ECIS2}\\

REDECAN - \cite{REDECAN2020}\\

Población INE - \cite{INEpob}\\

Defunciones Ministerio - \cite{MSCBS}

\subsection{Mortalidad por cáncer de hígado}

\subsection{Supervivencia de cáncer de hígado} 

\subsection{Prevalencia de cáncer de hígado}

% ---------------------------------

\section{Epidemiología del cáncer de colon-recto}

\subsection{Incidencia de cáncer de colon-recto}

\subsection{Mortalidad por cáncer de colon-recto}

\subsection{Supervivencia de cáncer de colon-recto} 

\subsection{Prevalencia de cáncer de colon-recto}






