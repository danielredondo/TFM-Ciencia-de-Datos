\chapter{Cáncer}

\section{¿Qué es el cáncer?}

El cáncer es un conjunto de enfermedades que tienen en común una división incontrolada de las células, que se propagan a tejidos circundantes \cite{NationalCancerInstitute2015}. Casi cualquier parte del cuerpo humano, que está formado por unas $3\cdot10^ {13}$ células \cite{Sender2016}, puede desarrollar cáncer. Por ello, aunque hablemos del cáncer como una única enfermedad, en realidad hay más de 100 tipos distintos de cáncer, con diferentes características.

\section{Causas del cáncer}

Las causas del cáncer son diversas, y el proceso de creación del cáncer es complejo y multifactorial. A menudo el causante del cáncer no es un solo elemento, sino la combinación e interacción de distintos factores ambientales y genéticos. Los factores ambientales son la causa del cáncer en un 90-95\% de los casos, mientras que la genética es responsable del restante 5-10\% \cite{SanchezPerez2016}, si bien estos porcentajes varían en función del tipo de cáncer.\\

Los factores causantes del cáncer se pueden clasificar principalmente en dos categorías: factores no modificables (esto es, factores sobre los que no tenemos ningún control) y factores modificables (donde se centra la prevención del cáncer).\\

\textbf{Factores no modificables:}
\begin{itemize}
	\item Edad. La edad es el principal factor de riesgo del cáncer. En general, el cáncer aumenta con la edad, aunque hay algunos cánceres que son más frecuentes en niños o adolescentes.
	\item Sexo. El cáncer suele ser más frecuente en hombres que en mujeres.
	\item Historial familiar, en algunas localizaciones anatómicas como mama, colon o próstata.
	\item Enfermedades genéticas, como el síndrome de Lynch y su relación con el cáncer colorrectal.
\end{itemize}

\textbf{Factores modificables:}
\begin{itemize}
	\item Tabaco. Aumenta el riesgo de incidencia de algún tipo de cáncer entre 10-20 veces más.
	\item Alcohol. El consumo de alcohol se asocia con una mayor incidencia de cáncer de localizaciones como estómago, mama, colon, o recto.
	\item Dieta, sendentarismo y peso corporal. Para prevenir el cáncer es importante llevar una dieta equilibrada, realizar ejercicio físico y evitar la obesidad.
	\item Exposición solar. Debe controlarse para disminuir el riesgo de melanoma y otros tumores de la piel.
	\item Exposición a distintos carcinógenos (agentes que producen cáncer). La \textit{International Agency for Research on Cancer} (IARC) tiene registrados 120 agentes (químicos, metales, radiaciones, productos farmacéuticos  ...) que  producen cáncer en humanos \cite{InternationalAgencyforResearchonCancer2019}, si bien diferentes exposiciones producen distintos tipos de cáncer \cite{Cogliano2011}.
\end{itemize}

\noindent Otros factores no se pueden clasificar como modificables o no modificables, ya que algunos de sus aspectos no se pueden cambiar. Los principales son:

\begin{itemize}
	\item Factores socioeconómicos. Son aquellas características del entorno del sujeto, como nivel socioeconómico, grado de cobertura sanitaria y lugar de residencia. Las desigualdades socioeconómicas se reflejan a menudo en desigualdades en cáncer.
	\item Factores reproductivos y hormonales, como la toma de anticonceptivos orales, la lactancia, o la terapia hormonal sustitutiva en mujeres menopáusicas.
\end{itemize}

\section{Impacto del cáncer}

El cáncer es en la actualidad uno de los mayores problemas de Salud Pública a nivel mundial. En el año 2018 se diagnosticaron en el mundo 18,1 millones de casos nuevos de cáncer, y se produjeron 9,6 millones de defunciones por cáncer \cite{Bray2018}. El cáncer de pulmón es el más diagnosticado (12\% del total), y la principal causa de muerte por cáncer (18\%). Existen importantes variaciones geográficas en la distribución del cáncer en el mundo, debidas principalmente a las diferencias en estilos de vida y al contexto económico y sanitario de las regiones \cite{Bray2018}.\\

En Europa, durante el año 2018 se han diagnosticado 3,9 millones de casos de cáncer (exceptuando el cáncer de piel no melanoma) y se han producido 1,9 millones de defunciones por cáncer \cite{ECIS2019}.\\

Los últimos datos de incidencia disponibles para España son del año 2019, donde se estima que más de 277.000 personas sean diagnosticadas de cáncer \cite{Galceran2019}, mientras que hubo más de 113.000 defunciones por cáncer en el año 2017 \cite{INEdef2019}.\\

A la vista de estos datos, podemos afirmar que el estudio del cáncer en todas sus vertientes es necesario para reducir el impacto de la enfermedad en la población, mediante acciones como la planificación de la atención sanitaria, la evaluación de la efectividad de sus tratamientos, o el desarrollo de estrategias de prevención como los programas de cribado.\\


\section{Cáncer de páncreas}
