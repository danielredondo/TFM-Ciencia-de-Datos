\chapter{\textit{Machine learning} aplicado a RNA-Seq}

\section{Selección de características}

\textcolor{red}{Ver apuntes asignatura bioinformática} \cite{HerreraMaldonado2020}.\\

\textcolor{red}{Leer y citar \cite{Xing}.}\\

\textcolor{red}{La selección de genes se ha utilizado ampliamente para intentar predecir el diagnóstico del cáncer, basándose en microarrays \cite{Lee2008, Maglietta2007}, aunque el uso de microarrays está siendo reemplazado por el uso de RNA-Seq.}\\

\subsection{Mínima redundancia, máxima relevancia (mRMR)}

\subsection{\textit{Random Forest} (RF)}

\subsection{Asociación de enfermedades (DA)}

\section{Algoritmos de clasificación}

\subsection{Máquinas de soporte vectorial (SVM)}

\subsection{k-vecinos más cercanos (kNN)}


