\chapter{\texttt{biomarkeRs}: una aplicación web interactiva para detección de biomarcadores}

\section{Desarrollo de la aplicación}

\textcolor{red}{\texttt{Shiny}, versión, documentación breve sobre Shiny, reactividad, incluye CSS, comentar que es accesible a todo el mundo (Shiny Server, o shinyapps.io), subir a una URL y compartir...}.\\

El fichero \texttt{shiny\textbackslash app.R} del repositorio de GitHub asociado al trabajo \cite{Redondo-Sanchez2020}  contiene el código de R desarrollado para crear la aplicación web.

\section{Utilidades de la aplicación}

\textcolor{red}{Útil para realizar análisis genéticos para personas sin apenas conocimientos previos de programación. Capturas de pantalla con ejemplos. Escribir pequeño manual de uso. Quizá grabar vídeo mostrando la aplicación (y subir GIF a README del repositorio de GitHub).}