\chapter{\texttt{biomarkeRs}: una aplicación web interactiva para detección de biomarcadores}

\section{Desarrollo de la aplicación}

Se ha realizado una aplicación web interactiva que realiza el proceso descrito. Se ha utilizado para ello el paquete de R \textit{\texttt{\{Shiny\}}} (v.1.2.0) \cite{Chang2020}, mejorando la interfaz usando CSS. La aplicación web puede ejecutarse de forma local usando RStudio (un entorno de desarrollo integrado de R) \cite{RStudioTeam2020} y también está disponible de forma online en el enlace \href{https://dredondo.shinyapps.io/biomarkeRs/}{\textcolor{blue}{https://dredondo.shinyapps.io/biomarkeRs/}}.\\

\textcolor{red}{Comentar el tipo de cuenta de shinyapps.io, cómo se ha subido a la web, es una especie de Docker, ...}

El fichero \texttt{shiny\textbackslash app.R} del repositorio de GitHub asociado al trabajo \cite{Redondo-Sanchez2020}  contiene el código de R desarrollado para crear la aplicación web.\\

\section{Utilidades de la aplicación}

\textcolor{red}{Útil para realizar análisis genéticos para personas sin apenas conocimientos previos de programación. Capturas de pantalla con ejemplos. Escribir pequeño manual de uso. Quizá grabar vídeo mostrando la aplicación (y subir GIF a README del repositorio de GitHub).}