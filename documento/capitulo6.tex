\chapter{Líneas abiertas de trabajo}

\section{Mejora de clasificación}

Para mejorar los resultados obtenidos en la clasificación de muestras de tejidos existen varias líneas abiertas de trabajo:

\begin{itemize}
	\item Otros métodos de selección de características. Para lograr un equilibrio entre resultados numéricos y clínicos, se pueden utilizar algoritmos de selección de características que usan evidencias existentes de relaciones gen-enfermedad y añaden reducción de la redundancia. Es el caso de DARED, un algoritmo en el que están trabajando los tutores de este trabajo y que supone una mejora con respecto al método DA.
	\item Otros métodos de clasificación. Además de los algoritmos de clasificación usados en el trabajo se pueden emplear otros métodos. Por ejemplo, se puede clasificar usando SVM con núcleos de base no radial o crear un \textit{ensemble} de algoritmos para permitir una estructura más flexible del modelo que quizá se adapte mejor a los datos.
	\item Combinar análisis de RNA con otros tipos de análisis. En el caso de la clasificación multiclase para colon-recto, donde los resultados pueden ser mejorables, se podría combinar el análisis de expresión de genes con análisis de microRNA o alteraciones somáticas.
\end{itemize}

\section{\{\texttt{KnowSeq}\}}

\{\texttt{KnowSeq}\} es un paquete de \texttt{R} en continua evolución y abierto a la colaboración mediante GitHub \cite{KnowSeq}. Algunas mejoras propuestas por el autor que se pueden implementar a corto plazo son las siguientes:

\begin{itemize}
	\item Un tuning preciso de los parámetros de los algoritmos de clasificación. En este trabajo, para realizar el tuning de los parámetros óptimos de SVM y kNN se han utilizado los 10 genes más relevantes. Posteriormente, se seleccionaba como mejor modelo aquel que con menor número de genes obtenía el mejor F1-Score. Es probable que para ese número de genes, menor de 10, existan otros parámetros que puedan optimizar los resultados del algoritmo. Tras implementar este cambio, se podrían obtener parámetros óptimos para cada método de selección de características y número de genes.
	\item Nuevos gráficos. Dentro de la función \texttt{KnowSeq::dataPlot} se pueden implementar los mapas de calor mostrados en el presente trabajo (como la Figura 8) para medir los distintos indicadores (F1-Score, precisión, sensibilidad y especificidad) obtenidos con varios métodos de selección de características. En el caso de estas figuras, permiten detectar fácilmente los indicadores más altos debido a los colores empleados y al resaltado en negrita del valor más alto (realizado automáticamente con el código).
\end{itemize}

\section{\texttt{biomarkeRs}}

Tras la creación de la aplicación web \texttt{biomarkeRs}, se deben hacer esfuerzos continuados para acercar el análisis de datos transcriptómicos a usuarios sin conocimientos previos de programación. En este sentido, el uso de tutoriales o cursos cortos para enseñar a usar \texttt{biomarkeRs} pueden ser de utilidad. Por otra parte, la aplicación web debería aprovechar todo el potencial presente en \{\texttt{KnowSeq}\} y actualizarse con las nuevas funcionalidades que se implementen en el paquete.\\

La aplicación web \texttt{biomarkeRs} está en una etapa inicial y puede ir actualizándose con múltiples mejoras:

\begin{itemize}
	\item Incorporación de análisis multiclase. 
	\item Redacción en inglés de un manual de uso para fomentar el uso de la aplicación. Otra opción complementaria puede ser la grabación de vídeos explicando el funcionamiento de la aplicación con casos de uso.
	\item Inclusión del F1-Score obtenido en el apartado de validación del modelo. Aunque este indicador se puede calcular a partir de la matriz de confusión, que esté presente de manera directa facilitaría la experiencia de usuario.
	\item  Incorporación en el apartado de validación del modelo de una tabla interactiva en la que aparezcan el código identificativo del paciente, su clase observada y la clase predicha por el modelo. Esta característica puede ser relevante para personal médico ya que aporta valor a las predicciones individuales de cada paciente. Para añadir la característica, se está pendiente de actualizar las funciones de \textit{\{\texttt{KnowSeq}\}} que evalúan modelos en el conjunto de test para que posibiliten la devolución de información predicha para cada paciente de forma detallada.
	\item Carga de datos personalizados. En el apartado de carga de datos se podría permitir al usuario que cargase conjuntos de train y test personalizados.
\end{itemize}

El hecho de que sea una aplicación de código abierto puede ayudar en la implementación de estas mejoras.

\section{Artículo}

Se está desarrollando un artículo científico que sintetiza los resultados obtenidos en el presente trabajo y será enviado a una revista del campo de la bioinformática.