\chapter{Líneas abiertas de trabajo}

\section{El problema de clasificación}

Para el problema de clasificación de muestras de tejidos existen varias líneas abiertas de trabajo:

\begin{itemize}
	\item Otros métodos de clasificación. Además de los algoritmos de clasificación SVM, RF y kNN, se pueden emplear otros métodos. Por ejemplo, se puede clasificar usando SVM con núcleos de base no radial o crear un \textit{ensemble} de algoritmos para permitir una estructura más flexible del modelo que se adapte mejor a los datos.
	\item Otros métodos de selección de características. Para lograr un equilibrio entre resultados numéricos y clínicos, se pueden utilizar algoritmos de selección de características que usan evidencias existentes de relaciones gen-enfermedad y añaden reducción de la redundancia. Es el caso de DARED, un algoritmo en el que están trabajando los tutores de este trabajo y que supone una mejora con respecto al método DA empleado en el capítulo 4.
	\item En el caso de la clasificación multiclase para colon-recto, donde los resultados no eran buenos, se podría combinar el análisis de expresión de genes con análisis de microRNA o alteraciones somáticas para mejora.
\end{itemize}

\section{\{\texttt{KnowSeq}\}}

\{\texttt{KnowSeq}\} es un paquete de \texttt{R} en continua evolución y abierto a la colaboración mediante GitHub \cite{KnowSeq}. Algunas mejoras propuestas por el autor que se pueden implementar a corto plazo son las siguientes:

\begin{itemize}
	\item Un tuning preciso de los parámetros de los algoritmos de clasificación. En este trabajo, para realizar el tuning de los parámetros óptimos de SVM y kNN se han utilizado los 10 genes más relevantes. Posteriormente, se seleccionaba como mejor modelo aquel que con menor número de genes obtenía el mejor F1-Score. Es probable que para ese número de genes, menor de 10, existan otros parámetros que puedan optimizar los resultados del algoritmo. Tras implementar este cambio, se podrían obtener parámetros óptimos para cada método de selección de características y número de genes.
	\item Nuevos gráficos. Dentro de la función \texttt{KnowSeq::dataPlot} se pueden implementar los mapas de calor mostrados en el presente trabajo (como la Figura 8) para medir los distintos indicadores (F1-Score, precisión, sensibilidad y especificidad) obtenidos con varios métodos de selección de características. En el caso de estas figuras, permiten detectar fácilmente los indicadores más altos debido a los colores empleados y al resaltado en negrita del valor más alto (realizado automáticamente con el código).
\end{itemize}

\section{\texttt{biomarkeRs}}

Tras la creación de la aplicación web \texttt{biomarkeRs} se deben hacer esfuerzos para acercar el análisis de datos transcriptómicos a usuarios sin conocimientos previos de programación. En este sentido, el uso de tutoriales o cursos cortos para enseñar a usar \texttt{biomarkeRs} pueden ser de utilidad. Por otra parte, la aplicación web debería aprovechar todo el potencial presente en \{\texttt{KnowSeq}\} y actualizarse con las nuevas funcionalidades que se implementen en el paquete.\\

La aplicación web \texttt{biomarkeRs} está aún en una etapa inicial, y se pueden añadir muchas mejoras:

\begin{itemize}
	\item Ampliar el propósito de la aplicación para incluir también análisis de tipo multiclase. 
	\item Redactar en inglés un manual de uso para fomentar el uso de la aplicación. Otra opción complementaria puede ser la grabación de vídeos explicando el funcionamiento de la aplicación con casos de uso.
	\item Añadir el F1-Score obtenido en el apartado de validación del modelo. Aunque este indicador se puede calcular a partir de la matriz de confusión, esto facilitaría la experiencia de usuario.
	\item \textcolor{red}{Añadir como cosas a mejorar de la app: que el usuario pueda elegir los IDs que se usan para train y test}
	\item\textcolor{red}{Se podría añadir en la tabla final de resultados una tabla interactiva con los IDs, su clase original, y su clase predicha por el modelo, con otra columna con "true positive/false positive/etc"... pero habría que modificar las funciones test de KnowSeq para que devuelva un vector con las clases predichas (en la actualidad, devuelve matriz de confusión, vectores de acc, spe, sens y f1). Pero hay que cambiar todas las funciones de test para que devuelvan las etiquetas predichas}
\end{itemize}

El hecho de que sea una aplicación de código abierto abierta a colaboraciones puede ayudar a implementar estas mejoras.

\section{Artículo}

Se está desarrollando un artículo científico que sintetiza los resultados obtenidos en el presente trabajo y será enviado a una revista del campo de la bioinformática.