	 % Agradecimientos
\null\newpage
\thispagestyle{empty}

\small{	 
	\textit{I'm a struggling comedian}
	
	\textit{I can't find the laughs}
	
	\textit{I'm a poor mathematician }
	
	\textit{I do things by halves}
	
	Keaton Henson - \textit{Career day}.
}

\vspace{1.4cm}

\noindent {\large\textbf{Agradecimientos:}}\\

Este Trabajo Fin de Máster se ha desarrollado en un contexto pandémico que aporta aún más valor a las personas a las que quiero dar las gracias. Todas ellas me han ayudado a conseguir la mejor versión de mí mismo.\\

En primer lugar, estoy muy agradecido a mis tutores Daniel y Luis Javier, que han sabido orientar mis esfuerzos con gran acierto. Su disponibilidad y cercanía pese a la distancia han facilitado mucho el trabajo, que ha mejorado notablemente con sus revisiones y comentarios. Agradezco también a Ignacio Rojas su tiempo y dedicación, ya que sus ideas aportadas en las reuniones iniciales fueron muy útiles para definir las líneas generales del presente trabajo. 
\\

Estoy profundamente agradecido al excelente profesorado que he tenido a lo largo de mi vida, a la educación pública y a las becas que me han permitido completar mis estudios.
\\

A todo el equipo del Registro de Cáncer de Granada le estoy agradecido por su inmensa influencia tanto en mi formación profesional como en mi evolución personal. El papel protagonista del cáncer en mi familia ha sido el principal combustible que ha impulsado mis ganas por investigar en este campo, en el que ahora estoy rodeado de excelentes profesionales.
\\

Este trabajo, así como muchos otros, no habrían sido posibles sin el apoyo diario incondicional de Sandra. Gracias a su gran sacrificio no he fallecido en el intento de compaginar trabajo, estudios, y vida personal. Debo agradecer también a mis familiares y amigos su comprensión con mi poco tiempo disponible.
\\

Quiero dedicar este trabajo a mi tita Inma y a mi hermano David, a mis padres y a mi hermana, a mis abuelas, a todos los pacientes oncológicos y sus familias. Como dice una magnífica investigadora a la que admiro: “el simple hecho de vivir provoca cáncer”. Vivamos. 

	\thispagestyle{empty}