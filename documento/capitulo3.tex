\chapter{\textit{Machine learning} aplicado a transcriptómica}

\section{Algoritmos de selección de características}

Los algoritmos de selección de características, o variables, consisten en la elección de un subconjunto de variables relevantes que permitan:

\begin{itemize}
	\item Obtener predicciones óptimas para un número bajo de genes.
	\item Proporcionar predictores menos costosos computacionalmente.
	\item Mejorar la interpretabilidad de los modelos resultantes, y facilitar la visualización de datos.
\end{itemize} 

La selección de características cuenta con interesantes aplicaciones en la genómica ya que permite encontrar conjuntos pequeños de biomarcadores que permiten diferenciar con gran precisión entre distintos estados o patologías \cite{Xing, Tadist2019}. Si se considera cada gen como una variable, la selección de características consigue reducir los problemas asociados a la maldición de la dimensionalidad \cite{Bellman1957, Bellman1961}. \\

Se distinguen principalmente 3 tipos de algoritmos de selección de características, descritos en varias referencias \cite{HerreraMaldonado2020, Tadist2019}:

\begin{itemize}
	\item Algoritmos de selección por filtrado. Utilizan técnicas estadísticas para identificar las variables más relevantes antes de diseñar el modelo predictivo. Suelen estar basados en medidas de correlación entre variables, como la información mutua, y pueden devolver un ranking de relevancia de las variables, o un subconjunto óptimo de variables. Entre sus ventajas destacan un bajo costo computacional en el entrenamiento del modelo, gran interpretabilidad y facilidad de implementación. Un ejemplo de algoritmo de selección por filtrado es mRMR (mínima redundancia, máxima relevancia). 
	
	\item Algoritmos de selección embebidos. Utilizan el método de entrenamiento del modelo para seleccionar simultáneamente las características más relevantes. Ejemplos de algoritmos de selección embebido son el uso de \textit{random forest} o máquinas de soporte vectorial. 
	
	\item Algoritmos de selección por envoltura. En estos métodos, el algoritmo de selección de variables está incluido en el propio modelo predictivo y es retroalimentado por él, seleccionando aquel modelo que proporciona mejor efectividad. El principal inconveniente de estos métodos es el elevado coste computacional, aunque como ventaja asegura el mejor rendimiento de entre todas las opciones que se han evaluado. Un ejemplo es MINT, una mejora de mRMR \cite{He2016}.
\end{itemize}

Se presentan a continuación los algoritmos de selección de variables que se utilizarán en el capítulo 4.

\subsection{Mínima redundancia máxima relevancia (mRMR)}

El método de mínima redundancia máxima relevancia (mRMR) está basado en el concepto de ``información mutua'' \cite{Koller1996}. La información mutua de dos variables se cuantifica como la reducción de incertidumbre sobre una de las variables conocida la otra.\\

El algoritmo mRMR funciona hacia delante: partiendo del conjunto vacío de características, selecciona aquella variable que tenga alta relevancia (su información mutua con la variable resultado sea mayor) pero a su vez tenga baja redundancia (información mutua) con el resto de variables ya seleccionadas \cite{HanchuanPeng2005}. Matemáticamente, en cada paso se selecciona la variable $X$ que maximiza la siguiente función: $$I(X,Y) - \dfrac{1}{\rvert S \rvert} \sum_{W\in S} I(X, W)$$

siendo $I$ la función que mide la información mutua entre dos variables, $Y$ la variable resultado y $S$ el conjunto de variables ya seleccionadas. El proceso se repite hasta alcanzar un cierto número prefijado de variables seleccionadas. El resultado final es un ranking de variables ordenadas en base a su importancia respecto al criterio mRMR.\\

Es un método muy conocido que ha sido utilizado ampliamente en ciencias -ómicas \cite{Ding2005, Yang2013, Galvez2018, Castillo2019, Galvez2020}. En \texttt{R}  está implementado mediante la función \texttt{KnowSeq::\linebreak featureSelection(mode = `mrmr')} \cite{KnowSeq}, que a su vez utiliza \texttt{praznik::MRMR} \cite{Kursa2020}.

\subsection{\textit{Random forest} (RF)}

Uno de los resultados del modelo de clasificación \textit{random forest}, detallado en la sección 3.2.2. es un ranking de variables según su importancia. Este método de selección de variables se trata por tanto de un método embebido.\\

La importancia de una variable en el modelo se puede cuantificar usando la reducción media en precisión del modelo al aleatorizar los valores de la variable manteniendo su distribución \cite{Breiman2001, Breiman2002}. También se puede usar la reducción media de otras medidas de entropía como el índice de Gini \cite{Louppe2013}.\\

Este algoritmo en \texttt{R}  está implementado en \texttt{KnowSeq::featureSelection(mode = `rf')} \cite{KnowSeq} que a su vez utiliza \texttt{randomForest::randomForest} \cite{Liaw2002}.

\subsection{Asociación de enfermedades (DA)}

El método de selección de características mediante asociación de enfermedades (DA, por sus siglas en inglés: \textit{Disease Association}) permite encontrar aquellos genes que están asociados en la literatura científica con una determinada enfermedad, en base a evidencias tales como rutas metabólicas afectadas, expresión de RNA o fármacos. Utiliza para ello la plataforma \textit{targetValidation} de \textit{Open Targets} \cite{OpenTargets2020}, que contiene para cada gen una puntuación midiendo la asociación gen-enfermedad en el rango de 0 (no hay asociación) a 1 (la asociación es total). El método DA obtiene esas puntuaciones y las ordena, obteniendo un ranking de genes en base a su asociación con la enfermedad.\\

El método DA  está implementado en \texttt{R}  en \texttt{KnowSeq::featureSelection(mode = `da')} \cite{KnowSeq}, que utiliza a su vez la REST API de \textit{targetValidation} \cite{OpenTargets2020}.

\section{Algoritmos de clasificación}

Dado un conjunto de datos con varias variables, siendo una de ellas la clase, el problema de clasificación consiste en encontrar un modelo que prediga la clase basándose en el resto de variables. Se presentan a continuación los  algoritmos de clasificación que se utilizarán en el capítulo 4.

\subsection{Máquinas de soporte vectorial (SVM)}

Las máquinas de soporte vectorial (SVM) son uno de los algoritmos más populares de la última década debido a su alta eficacia en problemas con grandes cantidades de datos. Consiste en un caso particular de (\textit{kernel machines}), una familia de métodos en el que el espacio de características se transforma para ser representados en un espacio más complejo, de alta dimensión o incluso infinita. Para ello, utilizan \textit{kernels}: funciones que devuelven el producto escalar entre las transformaciones de dos argumentos, sin tener que especificar de forma explícita la transformación realizada. Resumidamente, en los algoritmos SVM la transformación de los datos en el nuevo espacio permite que las clases sean linealmente separables mediante un hiperplano que separa las clases dejando el máximo margen posible entre ellas \cite{Boser1992}. Aquellas instancias que limitan el margen de ese hiperplano son llamados vectores soporte. Aunque en un principio el algoritmo SVM fue ideado para problemas binarios \cite{Boser1992}, se ha generalizado para problemas multi-clase \cite{Duan2005}.\\

Uno de los \textit{kernel} más comunes para SVM es el de base radial, que viene determinado por dos parámetros: coste ($c$) y gamma ($\gamma$). El coste $c$ mide la permisividad que se permite a la existencia de clases mal clasificadas por el modelo, que podrían ser ejemplos extraños (\textit{outliers}). Un coste alto puede sobreajustar el modelo (\textit{overfitting}), mientras que un coste bajo puede reducir la precisión del modelo en el conjunto de entrenamiento. El parámetro $\gamma$ mide el nivel de influencia de cada vector soporte para la construcción del hiperplano. Para cada problema se pueden encontrar valores de coste y gamma que optimizan los resultados del algoritmo SVM con base radial, aunque debido a que la búsqueda exhaustiva suele ser costosa computacionalmente se suele realizar una búsqueda de los mejores parámetros dentro de unos posibles valores de $c$ y $\gamma$.\\

Los algoritmos SVM están implementados en varios paquetes de R, siendo el más común \{\texttt{e1071}\} \cite{Meyer2019}.

\subsection{\textit{Random Forest} (RF)}

\textit{Random forest} (RF) es uno de los algoritmos de machine learning más usados en la actualidad, y se puede aplicar a tareas de clasificación y regresión. Para clasificación es un método en el que se crean varios árboles de decisión decorrelados entre sí para elegir la clase más votada por los árboles como la clase predicha. Para conseguir esta ausencia de correlación, cada vez que se considera una división en cada árbol se obliga a que la variable que dividirá las instancias tenga que pertenecer a un subconjunto de las variables seleccionado aleatoriamente \cite{Breiman2001, Breiman2002}. Debido a este método de construcción de árboles, es un algoritmo cuya principal desventaja es la ausencia de interpretabilidad.\\

El algoritmo de RF es una mejora del método de \textit{bagging} en árboles de decisión, que consiste en crear árboles basados en una selección aleatoria sin reemplazamiento de las instancias del conjunto de entrenamiento, reduciendo así la varianza de las predicciones \cite{Breiman1996}.\\

El paquete de \texttt{R} \{\texttt{randomForest}\} implementa el algoritmo RF \cite{Liaw2002}.

\subsection{k-vecinos más cercanos (kNN)}

El algoritmo de los k-vecinos consiste en asignar a cada dato sin clasificar la clase más común entre sus k-vecinos más cercanos \cite{Altman1992}. Aunque la distancia euclídea es la más usual, se pueden utilizar otras distancias para el cálculo de los k-vecinos más cercanos \cite{Hu2016}, y es recomendable normalizar los atributos para que todos tengan el mismo peso. Además, contando con un conjunto de entrenamiento se puede hallar el parámetro $k$ óptimo para el problema.\\

En este trabajo se utilizará la implementación de kNN realizada en \{\texttt{caret}\} \cite{Kuhn2020}.

\subsection{Medidas de evaluación}

\textcolor{red}{precisión (\textit{accuracy}), sensibilidad, especificidad y F1-Score. Validación cruzada.}
