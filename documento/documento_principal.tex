\documentclass[a4paper, 12pt]{bookln9}
\usepackage[utf8]{inputenc}
\usepackage[spanish]{babel}
\decimalpoint % To use point as decimal point
\usepackage{enumerate}
\usepackage{graphicx}
\usepackage{hyperref}
\usepackage{cite}
\usepackage{textcomp}
\usepackage{subfig}
\usepackage{multirow}
\usepackage{ntheorem}
\usepackage{hyperref}
\usepackage[dvipsnames]{xcolor}
\usepackage{color}
\usepackage{float}
\hypersetup{
	colorlinks = true,
	linkcolor = darkblue,
	anchorcolor = darkblue,
	citecolor = darkblue,
	filecolor = darkblue,
	urlcolor = darkblue
}
\usepackage{amsmath}
\usepackage{amssymb}
\usepackage{lscape}
\usepackage{url}

 % Para cajas con código R
\usepackage{listings}
\lstset{basicstyle=\ttfamily, breaklines=true}
\lstset{ 
	language=R,                     % the language of the code
	basicstyle=\footnotesize\ttfamily, % the size of the fonts that are used for the code
	numbers=left,                   % where to put the line-numbers
	numberstyle=\footnotesize\color{gray},  % the style that is used for the line-numbers
	stepnumber=1,                   % the step between two line-numbers. If it is 1, each line
	% will be numbered
	numbersep=5pt,                  % how far the line-numbers are from the code
	backgroundcolor=\color{white},  % choose the background color. You must add \usepackage{color}
	showspaces=false,               % show spaces adding particular underscores
	showstringspaces=false,         % underline spaces within strings
	showtabs=false,                 % show tabs within strings adding particular underscores
	frame=single,                   % adds a frame around the code
	rulecolor=\color{black},        % if not set, the frame-color may be changed on line-breaks within not-black text (e.g. commens (green here))
	tabsize=2,                      % sets default tabsize to 2 spaces
	captionpos=b,                   % sets the caption-position to bottom
	breaklines=true,                % sets automatic line breaking
	breakatwhitespace=false,        % sets if automatic breaks should only happen at whitespace
	keywordstyle=\color{RoyalBlue},     % keyword style
	commentstyle=\color{Mulberry},      % comment style
	stringstyle=\color{ForestGreen},    % string literal style
	literate={á}{{\'a}}1
	{ã}{{\~a}}1
	{é}{{\'e}}1
	{ó}{{\'o}}1
	{í}{{\'i}}1
	{ñ}{{\~n}}1
	{Ñ}{{\~N}}1
	{¡}{{!`}}1
	{¿}{{?`}}1
	{ú}{{\'u}}1
	{Í}{{\'I}}1
	{Ó}{{\'O}}1
} 


\lstset{
	language = VBScript,
	literate={á}{{\'a}}1
	{ã}{{\~a}}1
	{é}{{\'e}}1
	{ó}{{\'o}}1
	{í}{{\'i}}1
	{ñ}{{\~n}}1
	{Ñ}{{\~N}}1
	{¡}{{!`}}1
	{¿}{{?`}}1
	{ú}{{\'u}}1
	{Í}{{\'I}}1
	{Ó}{{\'O}}1
} 


\usepackage{fancy} % cls file
\usepackage{mathfont} % cls file

\definecolor{darkblue}{rgb}{0.0, 0.0, 0.55}



\newtheorem*{defi}{\normalfont\fontfamily{phv}\fontsize{12}{17}\bfseries Definici\'on}[section]
\newtheorem*{lema}{\normalfont\fontfamily{phv}\fontsize{12}{17}\bfseries Lema}[section]
\newtheorem*{conj}{\normalfont\fontfamily{phv}\fontsize{12}{17}\bfseries Conjetura}[section]
\newtheorem*{coro}{\normalfont\fontfamily{phv}\fontsize{12}{17}\bfseries Corolario}[section]
\newtheorem*{car}{\normalfont\fontfamily{phv}\fontsize{12}{17}\bfseries Caracterización}[section]
\newtheorem{propos}{\normalfont\fontfamily{phv}\fontsize{12}{17}\bfseries Proposición}[section]
\newtheorem*{prop}{\normalfont\fontfamily{phv}\fontsize{12}{17}\bfseries Propiedades}[section]
\newtheorem{nota}{\normalfont\fontfamily{phv}\fontsize{12}{17}\bfseries Nota}[section]
\newtheorem*{ejemplo}{\normalfont\fontfamily{phv}\fontsize{12}{17}\bfseries Ejemplo}[section]
\newtheorem{teorema}{\normalfont\fontfamily{phv}\fontsize{12}{17}\bfseries Teorema}[section]
\newtheorem*{demostracion}{\sc Demostración} 

\renewcommand{\thefootnote}{\arabic{footnote}}
%\renewcommand{\labelitemi}{$\cdot$}
\renewcommand\labelitemi{-} % - en itemize
%\renewcommand{\ttdefault}{phv}

%%%%%%%%%%%%%%%%%%%%%%% MEDIDAS %%%%%%%%%%%%%%%%%%%%%%%%%%%%%%%

% \parindent      0 mm
 \topmargin      10 mm   %%%%%% 0mm
 \headsep        8 mm   %%%%%% 8mm
 \headheight     5 mm   %%%%%% 5mm
 \textheight     211 mm
 \footskip       10 mm
 \headrulewidth  0.7pt
 \oddsidemargin  5 mm
 \evensidemargin  29.4 mm
\evensidemargin 5 mm
\textwidth      145 mm
\renewcommand{\baselinestretch}{1.15}
%%%%%%%%%%%%%%%%%%%%%%%%%%%


\setlength\parindent{0pt} %% Eliminar indent (sangría) en todo el documento

\begin{document}

%%%%%%%%%%%%%%%%%% HEADINGS %%%%%%%%%%%%%%%%%%%%%%%%%%%%%%%%%%%
 \pagestyle{fancyplain}
 \lhead[\fancyplain{}{\small\bf\thepage}]{\fancyplain{}{\small\bf\rightmark}}
 \rhead[\fancyplain{}{\small\bf\leftmark}]{\fancyplain{}{\small\bf\thepage}}
 \cfoot[\fancyplain{\small\bf\thepage}{}]{\fancyplain{\small\bf\thepage}{}}
%%%%%%%%%%%%%%%%%%%%%%%%%%%%%%%%%%%%%%%%%%%%%%%%%%%%%%%%%%%%%%%	

\author{\textbf{Autor:}\vspace{-0.1cm}\\ Daniel Redondo Sánchez \vspace{0.2cm}\\
\textbf{Tutores:}\vspace{0.1cm}\\
Ignacio Rojas\\
Luis Javier Herrera\\
Daniel Castillo
% \newline \textsc{Universidad de Granada}
}
	
\title{
	\vspace{-4cm}
	\centering
	{\Large TRABAJO FIN DE MÁSTER\\} \vspace{0.3cm}
	{\large \textsc{Máster Universitario Oficial en Ciencia de Datos e Ingeniería de Computadores\\}}
		{\LARGE \textbf{\bfseries{Epidemiología y detección de biomarcadores en cáncer\\}}}
		\vspace{-0.75cm}
}


\date{\vspace{1cm}
	Granada, septiembre de 2020 \\
	\vspace{0.5cm}
	\includegraphics[height=2.5cm]{logos/ugr.png} \\ 
}
	
	\mainmatter
	\maketitle
	\thispagestyle{empty}
	
	%% 
	%% \null\newpage
	%% \thispagestyle{empty}
	%% \noindent \textbf{Agradecimientos:}\\
	%% 
	%% 
	%% \null\newpage
	%%
	%% 
	
	% Índice
	\tableofcontents
		
	% Abstract
	\thispagestyle{plain}

\markboth{Abstract}{Abstract}

\vspace{-30pt}

\section*{Resumen}
\addcontentsline{toc}{chapter}{Resumen / Abstract}

\textbf{Introducción:} El cáncer es uno de los mayores problemas de salud pública del mundo con más de 17 millones de casos nuevos y 9 millones de defunciones al año. \\

\textbf{Métodos:} Este trabajo se centra en el cáncer de hígado y el cáncer de colon-recto, describiendo sus principales indicadores epidemiológicos y usando \textit{machine learning} para analizar más de 1.100 muestras de RNA-Seq procedentes de pacientes de cáncer. Para clasificación biclase (tumor vs. tejido normal) y multiclase (varios tipos de tumor vs. tejido normal) se identifican los 10 genes más relevantes y se construyen modelos predictivos con SVM, \textit{random forest} y kNN con validación cruzada 5-fold.\\

\textbf{Resultados:} Los mejores clasificadores biclase para cada algoritmo son validados con excelentes resultados para cáncer de hígado (F1-Score en test: 99,5\%) y cáncer de colon-recto (F1-Score: 100\%). En los mejores modelos para clasificación multiclase se obtienen medidas de evaluación inferiores tanto en hígado (F1-Score: 91,8\%) como en colon-recto (F1-Score: 79,3\%).\\

Se ha desarrollado una aplicación web, \texttt{\textit{BiomarkeRs}}, que implementa análisis de transcriptómica y puede resultar de utilidad para personas sin conocimientos previos de programación.\\

\textbf{Conclusiones:} SVM, random forest y kNN obtienen resultados muy similares, y consiguen distinguir correctamente entre tejidos tumorales y sanos, con algunos problemas para distinguir entre diferentes tipos de cáncer. Es necesaria una validación externa e interpretaciones clínicas para establecer de forma clara una asociación gen-enfermedad.\\

\textbf{Palabras clave:} epidemiología, transcriptómica, cáncer de hígado, cáncer de colon-recto, RNA-Seq, machine learning, SVM, random forest, kNN.

\newpage
\thispagestyle{plain}

\section*{Abstract}

\textbf{Introduction:} Cancer is one of the world's largest public health problems with more than 17 million new cases and 9 million deaths every year. \\

\textbf{Methods:} This work focuses on liver cancer and colon-rectum cancer, describing their main epidemiological indicators and using machine learning to analyze more than 1,100 RNA-Seq samples from cancer patients. For binary (tumor vs. normal tissue) and multiclass (various tumor types vs. normal tissue) classification, the 10 most relevant genes are identified and predictive models are constructed with SVM, random forest and kNN with 5-fold cross-validation. \\

\textbf{Results:} The best binary classifiers are validated with excellent results for liver cancer (F1-Score in test: 99.5\%) and colon-rectum cancer (F1-Score: 100\%). Lower evaluation measures are obtained in the best models for multiclass classification, both in liver (F1-Score: 91.8\%) and in colon-rectum (F1-Score: 79.3\%). \\

A web application has been developed, \texttt {\textit{BiomarkeRs}}, that implements transcriptomic analysis and can be useful for people without previous knowledge of programming. \\

\textbf {Conclusions:} SVM, random forest and kNN obtained very similar results, and managed to correctly distinguish between tumoral and normal tissues with some troubles distinguishing between different types of cancer. External validation and clinical interpretations are necessary to clearly establish a gene-disease association.\\


\textbf{Keywords:} epidemiology, transcriptomics, liver cancer, colorectal cancer, RNA-Seq, machine learning, SVM, random forest, kNN.

\newpage
\thispagestyle{empty}


	
	% Capítulos
	
	% Introducción
	\chapter{Introducción}

\section{Objetivos del trabajo}

En el presente Trabajo Fin de Máster se analiza la epidemiología de los cánceres de hígado y colon-recto y se detectan genes que permiten identificar tumores.

\begin{itemize}
	\item En el capítulo 1, 
	\item En el capítulo 2,
	\item En el capítulo 3,
	\item En el capítulo 4,
	\item En el capítulo 5,
	\item En el capítulo 6,
\end{itemize}

% ---------------------------------

\section{Cáncer}

El cáncer es una enfermedad en la que se produce una división incontrolada de las células \cite{AmericanCancerSociety2015}. Aunque generalmente se habla del cáncer como una única enfermedad se trata en realidad de un conjunto de enfermedades, existiendo más de 100 tipos distintos de cáncer \cite{NationalCancerInstitute2015}.\\

El cáncer es una enfermedad genética, esto es, causada por cambios en los genes que controlan las funciones celulares \cite{NationalCancerInstitute2015}. En general, el proceso de creación del cáncer es complejo y multifactorial: a menudo el causante no es un solo elemento, sino la combinación e interacción de distintos factores ambientales y genéticos \cite{Migliore2012}.\\

Los factores causantes del cáncer se pueden clasificar principalmente en tres categorías:
\begin{enumerate}
	\item Factores no modificables. Son elementos que no se pueden cambiar, como la edad o la herencia genética \cite{WorldHealthOrganization2014, WorldHealthOrganization2020}.
	\item Factores modificables o prevenibles, como el tabaco, el alcohol, la dieta o la exposición a distintos carcinógenos \cite{Cogliano2011}.
	\item Otros factores. Algunas circunstancias no se corresponden a ninguna de las categorías anteriores ya que algunos de sus aspectos no se pueden cambiar. Es el caso de  factores socioeconómicos (como cobertura sanitaria en el lugar de residencia o privación económica) y factores reproductivos u hormonales (como toma de anticonceptivos, lactancia materna o terapia hormonal sustitutiva en mujeres menopáusicas) \cite{WorldHealthOrganization2020}.
\end{enumerate}

A continuación se introducen dos tipos de cáncer con los que se trabajará más adelante: el cáncer de hígado y el cáncer de colon-recto.

% ---------------------------------

\subsection{Cáncer de hígado}

El cáncer de hígado se corresponde con el código C22 de la Clasificación Internacional de Enfermedades, Décima Revisión, integrando las neoplasias malignas de hígado y vías biliares intrahepáticas \cite{ICD10, cie10es}.

\subsubsection{Anatomía y funciones del hígado}

El hígado es el órgano interno más grande y pesado del cuerpo humano, está situado en el cuadrante superior derecho del abdomen, debajo de las costillas, y está compuesto principalmente por dos lóbulos \cite{Abdel-Misih2010}.\\

\newpage
\textbf{Figura 1}. Anatomía del abdomen humano. Ilustración de Ties van Brussel.
\begin{center}
\includegraphics[width=.70\textwidth]{figuras/anatomia_higado.png} \\
\end{center}

Las funciones del hígado son múltiples y diversas. Las principales son procesar, particionar y metabolizar macronutrientes, regular el volumen de sangre, apoyar al sistema inmune, eliminar sustancias químicas como el alcohol y otras drogas y producir bilis para absorber grasas \cite{Trefts2017}. Es un órgano imprescindible para la vida.

\subsubsection{Factores de riesgo}

Uno de los factores de riesgo más comunes del cáncer de hígado es la presencia de cirrosis, o sustitución de células sanas de hígado por tejido cicatrizado. La cirrosis puede producirse por varias causas, siendo las más habituales el consumo excesivo de alcohol y la infección con el virus de la hepatitis B o C \cite{AmericanCancerSociety2019}. Otros factores de riesgo son el tabaco, la obesidad, padecer diabetes tipo II y consumir esteroides anabólicos \cite{AmericanCancerSociety2019, Marrero2005}.\\

La prevención del cáncer de hígado se basa en reducir la exposición a factores de riesgo como el tabaco y el alcohol, y en vacunarse contra la hepatitis B \cite{AmericanCancerSociety2019}.

% ---------------------------------

\subsection{Cáncer de colon-recto}

Las neoplasias malignas de colon, recto, unión rectosigmoidea, ano y canal anal (códigos C18-C21 según la Clasificación Internacional de Enfermedades, Décima Revisión \cite{ICD10, cie10es}) a menudo se estudian agrupadas por tener características muy similares.

\subsubsection{Anatomía y funciones del colon-recto}

El colon tiene 3 funciones principales: absorción de agua y electrolitos, producción y absorción de vitaminas y movimiento de heces hacia el recto para su eliminación por el ano \cite{Azzouz2020}.\\

\textbf{Figura 2}. Anatomía del intestino humano. Ilustración de Ties van Brussel.
\begin{center}
	\includegraphics[width=.70\textwidth]{figuras/anatomia_cr.png} \\
\end{center}

\subsubsection{Factores de riesgo}

Entre los factores de riesgo del cáncer de colon-recto se puede distinguir entre factores modificables y no modificables.\\

Entre los factores de riesgo que son modificables destacan el sobrepeso, la inactividad física, las dietas con alto consumo de carnes rojas o procesadas, y el consumo de tabaco y alcohol \cite{AmericanCancerSociety2020}.\\

Una edad superior a 50 años, padecer diabetes tipo 2 y tener antecedentes personales o familiares de cáncer de colon-recto, pólipos o enfermedad intestinal inflamatoria, como colitis ulcerosa y enfermedad de Crohn, son algunos de los factores de riesgo no modificables \cite{AmericanCancerSociety2020}. También existen algunos síndromes hereditarios como el síndrome de Lynch que aumentan las posibilidades de padecer cáncer de colon-recto \cite{Lynch2003}.\\

Para intentar prevenir el cáncer de colon-recto se deben cambiar aquellos factores que son modificables: realizar ejercicio, mantener una dieta saludable y evitar el consumo de tabaco y alcohol. Además, en los últimos años se están implementando programas de cribado de cáncer de colon-recto para detectar pólipos o diagnosticar el cáncer en etapas iniciales mediante análisis como pruebas de sangre oculta en heces o colonoscopias \cite{Levin2008}.\\

% ---------------------------------

\section{Ciencias -ómicas}

Se presenta a continuación una corta introducción a las ciencias -ómicas, con el objetivo de comprender los conceptos que se utilizarán más adelante.

\subsection{Algunas definiciones básicas}

\begin{itemize}
	\item Los seres vivos están hechos de células. En el núcleo de cada célula se encuentran los cromosomas, estructuras que almacenan el material genético del individuo. Las células humanas tienen 46 cromosomas: 23 heredados de la madre y 23 heredados del padre.
	\item La información genética se transporta mediante los ácidos nucleicos: ácido desoxirribonucleico (DNA, por sus siglas en inglés) y ácido ribonucleico (RNA, por sus siglas en inglés) \cite{Pierce2010}.
	\item En el DNA hay 4 tipos de bases nitrogenadas: A, C, G y T.
	\item En el RNA hay 4 tipos de bases nitrogenadas: A, C, G y U.
	\item Los cromosomas están formados por ácido desoxirribonucleico (DNA, por sus siglas en inglés), una molécula que codifica las instrucciones genéticas para la vida.
	\item Un gen es la región del DNA que codifica una proteína. Las proteínas son cadenas de aminoácidos unidos por enlaces peptídicos (enlaces entre el grupo amino y carboxilo).
	\item El genoma es la secuencia de nucleótidos que forman el ADN de un individuo.
	\item El ácido ribonucleico (RNA, por sus siglas en inglés) es un ácido nucleico formado por ribonucleótidos. 
\end{itemize}

\subsection{Genómica}

La genómica es la ciencia que estudia la composición, estructura y función de los genomas. Se dedica por tanto a estudiar cromosomas, mutaciones y variaciones tanto de nucleótidos concretos como de regiones del genoma.\\

No debe confundirse con la genética, que estudia los genes de manera individual.\\

El análisis GWAS (Genome-wide association study) es un ejemplo de análisis genómico.

\subsection{Transcriptómica}

La transcriptómica estudia el transcriptoma, esto es, el conjunto de RNA presente en una célula). El transcriptoma indica el nivel de expresión de genes en un determinado momento.\\

Los análisis de RNA-Seq y microRNA se enmarcan en el ámbito de la transcriptómica. \\

\subsection{Otras ciencias -ómicas}

La proteómica es la ciencia que estudia y caracteriza el proteoma (imagen dinámica de todas las proteínas expresadas).\\

\textcolor{red}{Metabolómica}.

\section{RNA-Seq}

Leer \cite{Stark2019, VanVerk2013, CastilloSecilla2020}.\\

Este trabajo se enmarca dentro de la transcriptómica, y está basado en datos obtenidos mediante RNA-Seq, técnica en la que se cuentan distintas lecturas de cada gen para ver si están sobreexpresados o infraexpresados, para finalmente comparar esas expresiones con una referencia (por ejemplo, enfermos contra sanos). Como sólo se realizan cuentas de la expresión de los genes, el RNA-Seq de un individuo no permite su identificación, por lo que los datos a menudo son accesibles de manera abierta.

	
	% Epidemiología del cáncer
	\chapter{Epidemiología del cáncer}

La Epidemiología se ha definido tradicionalmente como el estudio de la distribución y de los determinantes de la enfermedad en los seres humanos [1]. En un sentido más amplio, sin limitarse únicamente a la enfermedad, la epidemiología se define como el estudio de la aparición y distribución de los estados o acontecimientos relacionados con la salud en poblaciones específicas, incluyendo el estudio de los determinantes de estos estados, y la aplicación de este conocimiento al control de los problemas de la salud [2, 3].\\

[1] MacMahon B, Pugh TF. Principios y métodos de Epidemiología. Segunda edición. México: La Prensa Médica Mexicana, 1983.
[2] Last JM. Diccionario de Epidemiología. Barcelona: Salvat, 1989.
[3] Porta M (editor). A Dictionary of Epidemiology (5th edition). New York: Oxford University Press; 2008.


\section{Indicadores epidemiológicos}

Para medir en la población el impacto del cáncer se utilizan principalmente cuatro indicadores:

\begin{itemize}
	\item \textbf{Incidencia} (casos nuevos). Mide el riesgo de presentar cáncer.
	\item \textbf{Mortalidad} (defunciones). Mide el riesgo de morir por cáncer.
	\item \textbf{Supervivencia} (porcentaje de casos vivos). Mide la historia natural del cáncer y efectividad del tratamiento.
	\item \textbf{Prevalencia} (casos nuevos y antiguos, vivos). Mide la carga asistencial de la enfermedad.
\end{itemize}

\textcolor{red}{Añadir tendencias}\\

% ---------------------------------

\textbf{Referencias:}

GLOBOCAN - \cite{Bray2018, GCO}

ECIS - \cite{ECIS, ECIS2}

REDECAN - \cite{REDECAN2020}

Población INE - \cite{INEpob}

Defunciones Ministerio - \cite{MSCBS}.

\section{Incidencia de cáncer}

Para medir de manera precisa la incidencia de cáncer en una población es necesaria la existencia de un Registro de Cáncer Poblacional. Estas entidades se dedican a registrar exhaustivamente todos los casos de cáncer diagnosticados en un área geográfica, y sus datos son muy útiles para todo tipo de estudios epidemiológicos. Algunos de estos Registros cubren la población de todo un país (por ejemplo, Canadá) mientras que otros cubren regiones concretas (por ejemplo, la provincia de Granada). Desgraciadamente, muchas áreas geográficas no están cubiertas por un Registro de Cáncer Poblacional. Es el caso de España, en el que sólo el 27\% de la población está cubierta por un Registro de Cáncer Poblacional \cite{Redondo-Sanchez2019}. Para conocer de manera estimada la incidencia de cáncer en territorios sin Registro de Cáncer Poblacional o proyectar la incidencia a años posteriores se utilizan diversos métodos matemáticos y estadísticos \cite{Bray2018, GCO, ECIS, ECIS2, REDECAN2020, Redondo-Sanchez2019}.\\

Con respecto a las medidas usadas para reportar la incidencia, la más sencilla y fácil de interpretar es el número nuevo de casos de cáncer, enmarcado siempre en un periodo concreto de tiempo y un área geográfica. A partir del número de casos se puede calcular la tasa bruta (TB), un indicador que tiene en cuenta el tamaño de la población \cite{IARC1995}.\\

$$\text{TB}  = \dfrac{\text{Número de casos nuevos}}{\text{Personas-año a riesgo}} \cdot 100.000 = \dfrac{N}{P} \cdot 100.000$$\\

$$\text{ASR} = 100.000 \cdot \sum_{i = 1}^{N} \omega_i \dfrac{N_i}{P_i} $$\\

\textcolor{red}{Tasa bruta. Tiene problemas por estructura de población. Tasas estandarizadas (mundiales, europeas viejas y nuevas). Se usan a veces otros indicadores como tasas acumulativas.}\\

Las poblaciones estándar más utilizadas son:
\begin{itemize}
	
	
	\item Antigua población estándar europea. Propuesta en 1976 \cite{Waterhouse1976} basándose en la estructura de edad de varias poblaciones escandinavas.
	
	\item Nueva población estándar europea. En el año 2013 se realiza una revisión de la población estándar europea de 1976 por parte de la Oficina Europea de Estadística (EUROSTAT) para que la población refleje fielmente el envejecimiento existente en la población europea \cite{EUROSTAT2013}. Debido a su novedad, el uso de esta población aún no está ampliamente extendido en los organismos internacionales \cite{ECIS2} y en ocasiones se reportan las dos tasas estandarizadas por las poblaciones estándar antigua y nueva \cite{ECIS}.
	
	\item Población mundial. Propuesta por primera vez en 1960 \cite{SegiM.1960} y modificada más tarde en 1966 \cite{Doll1966}, permite realizar comparaciones a nivel mundial.
\end{itemize}

\newpage
\textbf{Tabla 1}. Poblaciones estándar más frecuentes para el cálculo de tasas estandarizadas por edad.
\begin{table}[H]
	\begin{tabular}{|c|c|c|c|}
		\hline
		
		Grupo de edad  &  \begin{tabular}[c]{@{}c@{}}Población estándar\\ mundial\end{tabular}  &  \begin{tabular}[c]{@{}c@{}}Población estándar\\ europea 1976\end{tabular}  &  \begin{tabular}[c]{@{}c@{}}Población estándar\\ europea 2013\end{tabular}\\\hline
		
		0-4 años  &  12.000  &  8.000  &  5.000\\
		5-9 años  &  10.000  &  7.000  &  5.500\\
		10-14 años  &  9.000  &  7.000  &  5.500\\
		15-19 años  &  9.000  &  7.000  &  5.500\\
		20-24 años  &  8.000  &  7.000  &  6.000\\
		25-29 años  &  8.000  &  7.000  &  6.000\\
		30-34 años  &  6.000  &  7.000  &  6.500\\
		35-39 años  &  6.000  &  7.000  &  7.000\\
		40-44 años  &  6.000  &  7.000  &  7.000\\
		45-49 años  &  6.000  &  7.000  &  7.000\\
		50-54 años  &  5.000  &  7.000  &  7.000\\
		55-59 años  &  4.000  &  6.000  &  6.500\\
		60-64 años  &  4.000  &  5.000  &  6.000\\
		65-69 años  &  3.000  &  4.000  &  5.500\\
		70-74 años  &  2.000  &  3.000  &  5.000\\
		75-79 años  &  1.000  &  2.000  &  4.000\\
		80-84 años  &  500  &  1.000  &  2.500\\
		$\geq$85 años  &  500  &  1.000  &  2.500\\\hline
		
	\end{tabular}
\end{table}

Para utilizar notación internacional, la tasa estandarizada por la población mundial se notará ASR-W (\textit{Age-Standardised Rate, World standard population}), la tasa estandarizada por la población europea de 1976 se notará ASR-oE (\textit{old European standard population}) y la de 2013 se notará ASR-nE (\textit{new European standard population}).\\

\subsection{Incidencia del total del cáncer excepto piel no melanoma}

\textcolor{red}{Poca importancia de piel no melanoma}\\

\textcolor{red}{Diagrama de Marimekko de incidencia de cáncer. Añadir categoría de Otros}

\subsection{Incidencia de cáncer de hígado}

texto

\subsection{Incidencia de cáncer de colon-recto}

texto

\section{Mortalidad por cáncer}

\textcolor{red}{Cómo se obtiene la mortalidad. Importancia de certif de defunción. También estimaciones y proyecciones.}

\subsection{Mortalidad del total del cáncer excepto piel no melanoma}

texto

\subsection{Mortalidad de cáncer de hígado}

texto

\subsection{Mortalidad de cáncer de colon-recto}

texto

\section{Supervivencia de cáncer} 

\textcolor{red}{Supervivencia se calcula principalmente a partir de inc, mort y tablas de vida población general}

\subsection{Supervivencia del total del cáncer excepto piel no melanoma}

texto

\subsection{Supervivencia de cáncer de hígado}

texto

\subsection{Supervivencia de cáncer de colon-recto}

texto

\section{Prevalencia de cáncer}

texto

\subsection{Prevalencia del total del cáncer excepto piel no melanoma}

texto

\subsection{Prevalencia de cáncer de hígado}

texto

\subsection{Prevalencia de cáncer de colon-recto}

texto







	
	% Machine learning aplicado a transcriptómica
	\chapter{\textit{Machine learning} aplicado a transcriptómica}

\section{Algoritmos de selección de características}

\textcolor{red}{Ver apuntes asignatura bioinformática} \cite{HerreraMaldonado2020}.\\

\textcolor{red}{Leer y citar \cite{Xing}.}\\



\subsection{Mínima redundancia, máxima relevancia (mRMR)}

\subsection{\textit{Random Forest} (RF)}

\subsection{Asociación de enfermedades (DA)}

\section{Algoritmos de clasificación}

\subsection{Máquinas de soporte vectorial (SVM)}

\subsection{\textit{Random Forest} (RF)}

\subsection{k-vecinos más cercanos (kNN)}

\subsection{Medidas de evaluación: precisión (\textit{accuracy}) y F1-Score}



	
	% Detección de biomarcadores en cáncer de hígado y colon-recto
    \chapter{Detección de biomarcadores en cáncer de hígado}

\section{Introducción}

\section{Metodología}

\subsection{Fuente de datos}

La fuente de los datos es GDC (Genomic Data Commons) Portal \cite{GDCPortal}, una plataforma web sobre cáncer del Instituto Nacional del Cáncer de Estados Unidos (\textit{National Cancer Institute}) \cite{NationalCancerInstitute}. La plataforma GDC Portal fue desarrollada por el Instituto Nacional del Cáncer de Estados Unidos, la Universidad de Chicago, el Instituto de Ontario para la Investigación del Cáncer y la empresa \textit{Leidos Biomedical Research} \cite{Grossman2016}. Su principal fortaleza reside en la integración y armonización de diversas fuentes heterogéneas, creando así un sistema de información amplio y robusto. \\

\newpage
\textbf{\textcolor{red}{Figura XX}}. Diagrama de funcionalidad y utilidad de GDC. Extraído de Grossman et al. \cite{Grossman2016}.
\begin{center}
	\includegraphics[width=1\textwidth]{figuras/funcionamiento_gdc.jpeg} \\
\end{center}

GDC Portal, a día 22 de Junio, contenía información sobre unos 84.000 casos, 23.000 genes y más de 3 millones de mutaciones de genes \cite{GDCPortal}. Los datos de los que dispone son muy variados, y se pueden distinguir en tres grandes categorías:

\begin{itemize}
	\item Información clínica, como la edad del sujeto, su sexo o el estadio del cáncer del que ha sido diagnosticado.
	\item Información genética y transcriptómica proveniente de diversos proyectos de investigación.
	\item Imágenes de tejidos tumorales y sanos.
\end{itemize} 

Algunos de estos datos son abiertos, mientras que para otros es necesario solicitar acceso.

\subsection{Análisis}

Para el análisis se ha utilizado principalmente el paquete de R \texttt{KnowSeq} (v.1.2.0), librería que ha sido desarrollada por los tutores del presente trabajo, y en la que el autor ha contribuido con algunas nuevas funciones y pequeñas modificaciones \cite{KnowSeq}. El paquete está además disponible en Bioconductor, una relevante plataforma de código abierto en R para el análisis de datos en genómica y transcriptómica \cite{Gentleman2004}.\\

\textcolor{red}{Otros paquetes de R con versiones y referencias!}

\section{Resultados}

\section{Conclusiones}


	
	% Aplicación web para detección de biomarcadores
	\chapter{\texttt{biomarkeRs}: una aplicación web interactiva para detección de biomarcadores}

\section{Desarrollo de la aplicación}

Se ha realizado una aplicación web interactiva que realiza el proceso descrito. Se ha utilizado para ello el paquete de R \textit{\texttt{\{Shiny\}}} (v.1.2.0) \cite{Chang2020}, mejorando la interfaz usando CSS. La aplicación web puede ejecutarse de forma local usando RStudio (un entorno de desarrollo integrado de R) \cite{RStudioTeam2020} y también está disponible de forma online en el enlace \href{https://dredondo.shinyapps.io/biomarkeRs/}{\textcolor{blue}{https://dredondo.shinyapps.io/biomarkeRs/}}.\\

\textcolor{red}{Comentar el tipo de cuenta de shinyapps.io, cómo se ha subido a la web, es una especie de Docker, ...}

El fichero \texttt{shiny\textbackslash app.R} del repositorio de GitHub asociado al trabajo \cite{Redondo-Sanchez2020}  contiene el código de R desarrollado para crear la aplicación web.\\

\section{Utilidades de la aplicación}

\textcolor{red}{Útil para realizar análisis genéticos para personas sin apenas conocimientos previos de programación. Capturas de pantalla con ejemplos. Escribir pequeño manual de uso. Quizá grabar vídeo mostrando la aplicación (y subir GIF a README del repositorio de GitHub).}
	
	% Conclusiones y líneas abiertas de trabajo
	\chapter{Conclusiones y líneas abiertas de trabajo}

\textcolor{red}{Añadir otros métodos de selección de características como daRed (que tiene en cuenta información redundante entre genes) y otros métodos de clasificación (p.e. SVM con otros núcleos)}\\

\textcolor{red}{\texttt{KnowSeqWeb}, interfaz web para acercar el uso de KnowSeq a personas sin conocimientos de programación: personal médico, etc. Resaltar diferencias entre KnowSeqWeb y biomaRcadores: multiclase/biclase, sin preprocesamiento/con, proceso automático/guiado con más capacidad de decisión, ...}\\

\textcolor{red}{Más mejoras de KnowSeq}




	
	% Bibliografía
	\newpage
	\phantomsection
	\addcontentsline{toc}{chapter}{Bibliografía}
	\bibliographystyle{unsrt}
	\thispagestyle{empty}
	\bibliography{bibliografia}
	
	% Apéndices
	\appendix
	\backmatter
	\chapter{Anexo I: Artículo}
	\section*{Anexo I: Artículo científico}\label{anexo1}

\textcolor{red}{Escribir en Agosto}

		
\end{document}